\chapter{How to}

\section{Classes \& méthodes}
Pour les applications utilisant RMI, le portage vers RMI Lite ne bouleversera pas l'implémentation déjà existante à condition de se limiter aux fonctionnalités disponibles. En effet, on peut faire un parallèle entre les classes de RMI Lite et celles de java.rmi :
\begin{itemize}
\item RemoteObjectProvider = java.rmi.server.UnicastRemoteObject
\item NamingServer = java.rmi.registry.LocateRegistry
\item Registry = java.rmi.registry.Registry
\end{itemize}
\medskip

Une fois les correspondances connues, les noms des méthodes restent sensiblement les mêmes... A noter qu'il n'est pas possible de créer automatiquement un stub en héritant de la classe RemoteObjectProvider comme avec UnicastRemoteObject en java.rmi. La méthode exportObject est obligatoire.

\section{Configuration}
\hspace{-.6cm}Par défaut, deux configuration sont proposées au sein de l'API : 
\begin{itemize}
\item ConfigManager\_RMI : encapsule java.rmi.
\item ConfigManager\_Socket : utilise RMI Lite avec java.io pour la communication.
\end{itemize}
\medskip
Il est aussi possible de configurer le middleware :
\begin{lstlisting}
Manager_IO io = new Manager_IO(); // couche de communication
NamingServer_Socket ns;
ns = new NamingServer_Socket();
RemoteObjectProvider_Socket rop;
rop = new RemoteObjectProvider_Socket();

rop.setIOManager(io);
io.setRemoteObjectManager(rop);
ns.setRemoteObjectProvider(rop);

RemoteMethodFactory.remoteObjectManager = rop; // usine d'appel distant
StubFactory.ioManager = io; // usine de stubs
\end{lstlisting}

