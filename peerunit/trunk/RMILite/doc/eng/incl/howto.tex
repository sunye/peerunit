\chapter{How to}

\section{Classes and methods}
For the applications already using RMI, migrating from RMI to RMILite should not raise problems if your use of RMI is limited to the features available in RMILite. See this equivalence between RMILite classes and RMI classes:
\begin{itemize}
\item RemoteObjectProvider = java.rmi.server.UnicastRemoteObject
\item NamingServer = java.rmi.registry.LocateRegistry
\item Registry = java.rmi.registry.Registry
\end{itemize}
\medskip
Once you know those classes, methods remain the same. Note that it's not possible to automatically create a stub by inherit by RemoteObjectProvider just like UnicastRemoteObject. The method exportObject is required.

\section{Configuration}
By default, two setups are proposed in the API: 
\begin{itemize}
\item ConfigManager\_RMI : encapsule java.rmi.
\item ConfigManager\_Socket : utilise RMI Lite avec java.io pour la communication.
\end{itemize}
\medskip
You can configure the middleware too :
\begin{lstlisting}
Manager_IO io = new Manager_IO(); // communication layer
NamingServer_Socket ns;
ns = new NamingServer_Socket();
RemoteObjectProvider_Socket rop;
rop = new RemoteObjectProvider_Socket();

rop.setIOManager(io);
io.setRemoteObjectManager(rop);
ns.setRemoteObjectProvider(rop);

RemoteMethodFactory.remoteObjectManager = rop; // distant call factory
StubFactory.ioManager = io; // stub factory
\end{lstlisting}

